\section{Background}
\label{sec:background}

The datasets we analyze are part of a U.S. Geological Survey (USGS) ScienceBase model archive summary report. These calibration data files contain observations collected at recreational Great Lakes beaches, including E.~coli concentrations and a range of environmental, meteorological, and biological factors that potentially influence bacterial levels. Furthermore, the data was collected in cooperation with regional health agencies and the U.S. Environmental Protection Agency’s Great Lakes Restoration Initiative. Discrete water-quality samples were collected weekly for five days per week from May–September 2019. Traditional culture-based methods require 18–24 hours for E.~coli results, motivating the use of rapid predictive models. The NowCast approach replaces persistence (previous-day
concentration) with site-specific multiple linear regression models developed in EPA’s Virtual Beach software \cite{USGS2021P9Y9O1YJ}. These models use easily measured environmental surrogates (such as turbidity, water temperature, rainfall, and lake level change) to estimate E.~coli concentration or exceedance probability in near real-time, enabling daily public advisories for recreational sites.


\subsection{Features}
We summarize the broad categories of features typically collected for NowCast model calibration datasets. These features fall into three main categories: Environmental Conditions, Meteorological Variables, and Observational/Biological Variables.

\textbf{Environmental Conditions}: 
\begin{itemize}
  \item \textbf{RHUM\_PCT:} Relative humidity (\%).
  \item \textbf{WTEMP\_CEL:} Water temperature ($^\circ$C).
  \item \textbf{CHANGELL\_FT:} Lake level change over the last 24 hours (ft).
\end{itemize}

\textbf{Meteorological Variables}:
\begin{itemize}
  \item \textbf{AirportWindSpInst\_mph:} Wind speed (mph).
  \item \textbf{AirportRain48W\_in:} Weighted 48-h rainfall (in).
\end{itemize}

\textbf{Observational/Biological Variables}:
\begin{itemize}
  \item \textbf{BIRDS\_NO:} Number of birds observed at the beach.
  \item \textbf{TURB\_NTRU:} Turbidity (NTRU); requires $\log_{10}$ transform.
\end{itemize}

\subsection{Comparative Features}
Although both datasets measure similar environmental conditions, the variable names and transformations differ slightly. Table \ref{tab:datasets} summarizes the predictors available at each site.

\begin{table}[htbp]
\centering
\caption{Comparison of predictor variables and transformations across Beach~6 (Erie, PA) and Huntington (Cleveland, OH).}
\label{tab:datasets}
\renewcommand{\arraystretch}{1.5}
\resizebox{\linewidth}{!}{%
\begin{tabular}{lll}
\hline
\textbf{Category} & \textbf{Beach 6 (Beach6\_2019)} & \textbf{Huntington (Huntington\_2019)} \\
\hline
Target & LAB\_ECOLI $\rightarrow \log_{10}$ & EcoliAve\_CFU $\rightarrow \log_{10}$ \\
Environmental & WTEMP\_CEL (water temp, °C) & Lake\_Temp\_C (water temp, °C) \\
 & CHANGELL\_FT (lake level change, ft) & LL\_PreDay (lake level change, ft) \\
 & TURB\_NTRU (NTRU) $\rightarrow \log_{10}$ & Lake\_Turb\_NTRU (NTRU) $\rightarrow \log_{10}$ \\
 & --- & WaveHt\_Ft (ft) $\rightarrow \sqrt{\cdot}$ \\
Meteorological & RHUM\_PCT (\%) & --- \\
 & AirportWindSpInst\_mph (mph) & --- \\
 & AirportRain48W\_in (in) & AirportRain48W\_in (in) \\
Observational & BIRDS\_NO (bird counts) & --- \\
\hline
\end{tabular}%
}
\end{table}