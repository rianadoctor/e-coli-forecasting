
% The \section{} command formats and sets the title of this
% section. We'll deal with labels later.
\section{Introduction}
\label{sec:intro}
In the Great Lakes region, elevated levels of E.coli can pose serious risks to swimmers and require quick advisories from public health officials, making accurate and timely estimates of these E.coli concentrations critical. Traditional water testing methods don't "accurately reflect current water-quality conditions", because results with these methods can only be obtained 18 to 24 hours post-sampling \cite{USGS2021P9Y9O1YJ}. To address these inaccuracies, USGS and collaborators developed predictive models that use environmental and water quality measurements to estimate E.coli concentrations quickly. In this paper, we reproduce those models using the calibration and validation datasets provided by the USGS and compare our results to their official models. We then evaluate how closely our reproduced models match the USGS benchmark in terms of fitted coefficients and predictive accuracy ($R^{2}$, Root Mean Squared Error (RMSE)). By comparing our models to the original, we are able to assess the reproducibility of the USGS approach as well as reflect on model quality and its broader impact. The dataset used comes from model archives, more specifically, the 2019 validation calibration files and validation data files were used, which are both published on ScieneBase.

% Citations: As you can see above, you create a citation by using the
% \cite{} command. Inside the braces, you provide a "key" that is
% uniue to the paper/book/resource you are citing. How do you
% associate a key with a specific paper? You do so in a separate bib
% file --- for this document, the bib file is called
% project1.bib. Open that file to continue reading...

% Note that merely hitting the "return" key will not start a new line
% in LaTeX. To break a line, you need to end it with \\. To begin a 
% new paragraph, end a line with \\, leave a blank
% line, and then start the next line (like in this example).

