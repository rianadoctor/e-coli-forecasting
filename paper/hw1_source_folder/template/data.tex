\section{Data}

We use two U.S. Geological Survey (USGS) calibration datasets
\cite{USGS2021P9Y9O1YJ} that support site-specific multiple linear
regression (MLR) ``NowCast'' models for predicting
\textit{Escherichia coli} (E.~coli) at Great Lakes beaches.

\subsection{Presque Isle Beach 6 (Pennsylvania)}
The \texttt{Beach6\_2019\_calibration\_data.csv} file contains E.~coli
samples collected at Presque Isle State Park, Erie, PA. Sampling
occurred May--September 2019, typically 5~days per week. Predictor
variables include turbidity, wave height, lake level, lake temperature,
and weighted 48-hour rainfall indices. The E.~coli response is modeled
as $\log_{10}$(E.~coli~CFU/100~mL).

\subsection{Huntington Reservation (Ohio)}
The \texttt{Huntington\_2019\_calibration\_data.csv} file contains
E.~coli samples from Huntington Reservation (Cleveland Metroparks, OH),
with three related sampling sites (Central, West, Composite).
Environmental covariates mirror those at Beach 6: turbidity, wave
height, lake level change, lake temperature, and weighted rainfall. The
same $\log_{10}$ transformation of E.~coli counts is applied.

\subsection{Variables and Transformations}
Table~\ref{tab:variables} summarizes the core predictors shared across
sites and their model-ready transformations.

\begin{table}[h]
\centering
\caption{Predictor variables used in NowCast calibration datasets.}
\label{tab:variables}
\begin{tabular}{lll}
\hline
\textbf{Variable} & \textbf{Units} & \textbf{Transformation} \\
\hline
E.~coli & CFU/100 mL & $\log_{10}$ response \\
Turbidity & NTRU & $\log_{10}$ \\
Wave height & ft & $\sqrt{\cdot}$ \\
Lake temperature & $^{\circ}$C & Linear \\
Lake level change & ft & Linear (signed) \\
Weighted rainfall (48-h) & in & $\sqrt{(2D_{m1}+D_{m2})}$ \\
\hline
\end{tabular}
\end{table}

\subsection{Notes}
\begin{itemize}
    \item Zero/nonpositive E.~coli values must be handled before
    $\log_{10}$ transformation.
    \item Predictors are aligned by sampling date; lake-level change
    uses the 8\,am reporting convention.
    \item The bathing-water threshold is 235~CFU/100~mL.
\end{itemize}
