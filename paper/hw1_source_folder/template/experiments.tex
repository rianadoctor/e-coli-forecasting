\section{Experiments}
\label{sec:expts}

For this project, we focus on two site-specific calibration datasets
released by the U.S. Geological Survey (USGS) in 2021
\cite{USGS2021P9Y9O1YJ}:

\begin{itemize}
  \item \texttt{Beach6\_2019\_calibration\_data.csv} (Presque Isle Beach~6, Erie, PA)
  \item \texttt{Huntington\_2019\_calibration\_data.csv} (Huntington Reservation, Cleveland Metroparks, OH)
\end{itemize}

Presque Isle Beach~6 corresponds to USGS station 420839080081801 in Erie, Pennsylvania. Huntington Reservation comprises three sampling sites in Cleveland, Ohio (Central: 412928081560220; West: 412929081561100; Composite: 412928081560215). In both cases, discrete E.~coli samples were paired with environmental measurements collected on-site and meteorological data compiled from NOAA’s National Centers for Environmental Information and Tides \& Currents programs.

The predictive models in the original USGS archive were developed using Virtual Beach version~3.07, with explanatory variables mathematically transformed to improve linearity (e.g., log$_{10}$ turbidity, square-root rainfall and wave height, untransformed lake temperature and lake-level change). Model performance was compared against the baseline persistence method, with results informing the public via the Great Lakes NowCast system.

The Beach~6 calibration dataset includes 463 water samples, while the Huntington calibration dataset contains 1,001 water samples, each spanning May--September 2019. In addition, the Beach~6 validation dataset includes 100 observations, while the Huntington validation data inlcudes 103 observations. Our target variable, \textbf{LAB\_ECOLI} (Beach~6) and \textbf{EcoliAve\_CFU} (Huntington), is transformed to $\log_{10}$(CFU/100 mL) for modeling. Both sites use a threshold of 2.37 $log_{10}$ CFU/100 mL as the water-quality exceedance benchmark.



