\section{Broader Impacts}
\label{sec:impacts}

Predictive beach ``nowcast'' models have clear public-health value by reducing the 24-hour culture lag in traditional E.~coli testing. When accurate, they can improve same-day advisories and potentially reduce illness \cite{Heasley2021,Guo2021,Searcy2018,Searcy2023}. However, these systems also raise equity considerations. Firstly, access to clean, open beaches is not evenly distributed. Studies document landscape-scale inequities in coastal access and amenities, with neighborhoods of color and lower-income communities often facing more barriers or reputational harms from recurrent closures \cite{Montgomery2015,Twichell2022}. Additionally, water-quality burdens and infrastructure gaps disproportionately affect socially vulnerable populations across the U.S. \cite{Gochfeld2011,Neville2022,USWaterAlliance2017}. So, considering that these communities may already face a lack of investment in infrastructure and monitoring, if predictive systems consistently flag certain beaches as "unsafe", then these areas risk being stigmatized. Communities may be labeled as "dirty" or unsafe for recreation. This can reinforce stereotypes tied to poverty or racial demographics, rather than reflecting inequities in environmental management. While these methods can provide critical real-time information, they also risk shifting responsibility away from systemic solutions and toward communities already burdened by environmental risk. Overall, predictive models can improve public health outcomes, but their deployment should be used with attention to equity, community, and trust. 



